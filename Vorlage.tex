% Dokumentenklasse: 
%   - {article} : für Kurzberichte.
% Dokumentenklasseoptionen:
%   - [11pt]    : Schriftgröße
%   - [a4paper] : Seitenformat DIN A4
%   - [twoside] : zweiseitig bedruckt
%   - [fleqn]   : linksbündige Ausrichtung der Gleichungen
\documentclass[11pt, a4paper, twoside, fleqn]{article}
\usepackage{ucs} %Γ Γιατί χρειζόμαστε αυτό το πακέτο;
% Das Paket inputenc erlaubt die direkte Eingabe von Sonderzeichen wie zum Beispiel deutschen Umlauten, um deren Trennung zu ermöglichen wird zudem das Paket fontenc mit eingebunden.
\usepackage[utf8x]{inputenc}
\usepackage[T1]{fontenc}
% Hierbei wird ngerman als Option des Paketes babel gesetzt.
\usepackage[ngerman]{babel}
% Mit fullpage kann die Texthöhe und -breite sowie die Ränder festlegen, dass die Seite fast voll ist.
\usepackage{fullpage}
\usepackage{amsmath, amssymb}
\usepackage{multicol}
%Γ Να προσθεθούν οι εντολές όπου ολο το LaTeX δεν έχει Indents και οι αποστάσεις μεταξύ των παραγράφων να είναι σταθερές όπως επίσης και οι αποστάσεις μεταξύ των σειρών.
% Mit xcolor können Seiten, die Schrift, Rahmen und Felder in den verfügbaren Farben gesetzt werden.
\usepackage{xcolor}
\usepackage{tikz}
% Anpassbare Enumerates/Itemizes
\usepackage{enumitem}
\setlength{\mathindent}{0cm}
\setlength\parindent{0pt}
\title{xxx} 
\author{Ioannis Christodoulakis}
\date{\today}
\begin{document}
% Der Befehl \selectlanguage{ngerman} ändert die Standardsprache des Dokumentes.
\selectlanguage{ngerman}
\maketitle
% Die Anweisung beendet lediglich die aktuelle Seite.
\newpage
% Dieser Befehl veranlasst LaTeX ein Inhaltsverzeichnis zu erzeugen. 
\tableofcontents
% Die Anweisung beendet lediglich die aktuelle Seite.
\newpage
%%%%%%%%%%%%%%%%%%%%%%%%%%%%%%%%%%%%%%%%%%%%%%%%%%%%%%%%%%%%%%%%%%%%%%%%%%%%%%%%%%%%%%%%%%%%%%%%%%%%
%%%%%%%%%%%%%%%%%%%%%%%%%%%%%%%%%%%%%%%%%%%%%%%%%%%%%%%%%%%%%%%%%%%%%%%%%%%%%%%%%%%%%%%%%%%%%%%%%%%%
\section{xxx}

\end{document}
