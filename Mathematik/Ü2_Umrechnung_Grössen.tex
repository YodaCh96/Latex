% Dokumentenklasse: 
%   - {article} : für Kurzberichte.
% Dokumentenklasseoptionen:
%   - [11pt]    : Schriftgröße
%   - [a4paper] : Seitenformat DIN A4
%   - [twoside] : zweiseitig bedruckt
%   - [fleqn]   : linksbündige Ausrichtung der Gleichungen
\documentclass[11pt, a4paper, twoside, fleqn]{article}
% Das Paket inputenc erlaubt die direkte Eingabe von Sonderzeichen wie zum Beispiel deutschen Umlauten,
%  um deren Trennung zu ermöglichen wird zudem das Paket fontenc mit eingebunden.
\usepackage[utf8]{inputenc}
\usepackage[T1]{fontenc}
% Hierbei wird ngerman als Option des Paketes babel gesetzt.
\usepackage[ngerman]{babel}
% Mit fullpage kann die Texthöhe und -breite sowie die Ränder festlegen, dass die Seite fast voll ist.
\usepackage{fullpage}
\usepackage{amsmath, amssymb}
\usepackage{multicol}
% With xcolor, pages, the font, frames and fields can be set in the available colors.
\usepackage{xcolor}
\usepackage{tikz}
% Customizable Enumerates/Itemizes
\usepackage{enumitem}
\setlength{\mathindent}{0cm}
\setlength\parindent{0pt}
\setlength{\parskip}{0em}
\renewcommand{\baselinestretch}{1.5}
\title{\Huge Mathematik 1\\
\large Umrechnung Größen}
\author{Ioannis Christodoulakis}
\date{\today}
\begin{document}
% The command \selectlanguage{ngerman} changes the default language of the document.
\selectlanguage{ngerman}
\maketitle
% The statement ends the current page.
\newpage
% This command causes LaTeX to create a table of contents.
\tableofcontents
% The statement ends the current page.
\newpage
%%%%%%%%%%%%%%%%%%%%%%%%%%%%%%%%%%%%%%%%%%%%%%%%%%%%%%%%%%%%%%%%%%%%%%
\section{Übung3: Umrechnung von Größen mit Einheiten}
\begin{enumerate}[itemsep=1ex, leftmargin=*]
\item Längen
    \begin{enumerate}[itemsep=2mm]
        \item[a)] $ 6820 mm = \textcolor{red}{6.82 m} $
        \item[b)] $ 1.044 km = \textcolor{red}{1044 m} $
        \item[c)] $ 580 \mu m = \textcolor{red}{0.58 mm} $
        \item[d)] $ 6.65 \cdot 10^{-7} m = \textcolor{red}{665 nm} $
        \item[e)] $ 6378 km = \textcolor{red}{6.378 \cdot 10^{6} m } $ (Exp. Darstellung)
        \item[f)] $ 8.7 LJ = \textcolor{red}{8.2 \cdot 10^{16} m} $ (Exp. Darstellung)
    \end{enumerate}
    Anmerkung. LJ = Lichtjahr ; Lichtgeschw. $ c = 3\cdot 10^{8} m/s : $
\item Flächen
    \begin{enumerate}[itemsep=2mm]
        \item[a)] $ 1690 mm^{2} = \textcolor{red}{0.169 dm^{2}} $
        \item[b)] $ 0.045 m^{2} = \textcolor{red}{450 cm^{2}} $
        \item[c)] $ 1.83 km^{2} = \textcolor{red}{1.83 \cdot 10^{6} m^{2}} $
        \item[d)] $ 79.1 dm^{2} = \textcolor{red}{0.791 m^{2}} $
        \item[e)] $ 10 \mu m \cdot 4 \mu m = \textcolor{red}{4 \cdot 10^{-11} m^{2}} $ 
        \item[f)] $ 4 \cdot \pi \cdot (6378 km)^{2} = \textcolor{red}{5.1 \cdot 10^{14} m^{2}} $
    \end{enumerate}
\item Volumen 
    \begin{enumerate}[itemsep=2mm]
        \item[a)] $ 92'500 mm^{3} = \textcolor{red}{92.5 cm^{3} = 0.0925 dm^{3}} $
        \item[b)] $ 65.8 dm^{3} = \textcolor{red}{0.0658 m^{3}} $
        \item[c)] $ 34.5 hl = \textcolor{red}{3450 dm^{3}} $
        \item[d)] $ 0.0285 m^{3} = \textcolor{red}{28.5 l} $ (l = Liter)
        \item[e)] $ 387'000 l = \textcolor{red}{3.87 \cdot 10^{-7} km^{3}} $
        \item[f)] $ 25.3 ml = \textcolor{red}{25.3 cm^{3} = 0.0253 dm^{3}} $
    \end{enumerate}
\newpage
\item Zeit und Winkel
 \begin{enumerate}[itemsep=2mm]
     \item[a)] $ 14800 s = $ \textcolor{red}{4h 6min 40s}
     \item[b)] $ 0.615 y = \textcolor{red}{1.94 \cdot10^{7} s} $
     \item[c)] $ 270 \mu s =  \textcolor{red}{0.27 ms} $
     \item[d)] 14 h 22 min 38 s = $\textcolor{red}{1.64 \cdot 10^{-3} y} $
     \item[e)] $ 14.33 ^{\circ} = \textcolor{red}{14 ^{\circ} 19' \cdot 48''}$ 
    \item[f)] $ 5 ^{\circ} 20' 35'' = \textcolor{red}{5,34 ^{\circ}} $
 \end{enumerate}
 \item Zusammengesetze Größen
    \begin{enumerate}[itemsep=2mm]
        \item [a)] $ 60 km/h = \textcolor{red}{16.67 m/s} $
        \item[b)] $ 7400 hPa = \textcolor{red}{74 N/cm^{2}} $ (1 Pa = $1 N/m^{2}) $ 
        \item[c)] $ 7.87 g/cm^{3} = \textcolor{red}{7.87 kg/dm^{3}} $
        \item[d)] $ 35 m/s = \textcolor{red}{126 km/h} $
        \item[e)] $ 28.5 l/min = \textcolor{red}{1.71 m^{3} / h} $
        \item[f)] $ 15 ^{\circ} /h = \textcolor{red}{0.25 ^{\circ} / min} $
    \end{enumerate}
\item Berechnungen mit gemischten Einheiten (anspruchsvoll)
    \begin{enumerate}[itemsep=2mm]
        \item[a)] $ \dfrac{1dl}{\frac{4}{3} \cdot \pi \cdot (1.5 mm)^{3}} = \textcolor{red}{7073} $
        \item[b)] $ \dfrac{150 Mio. km}{3 \cdot 10^{8} m/s} = \textcolor{red}{500 s} $
        \item[c)] $ \dfrac{4}{3} \cdot \pi \cdot (0.125 m)^{3} \cdot 2.7 \dfrac{g}{cm^{3}} = \textcolor{red}{22.1 kg} $
        \item[d)] $ 0.85 \dfrac{g}{cm^{3}} \cdot 4.5 \dfrac{m}{s} \cdot (3.5 cm)^{2} \cdot \pi \cdot 15 min = \textcolor{red}{13248kg} $
        \item[e)] $ 6 dm^{3} \cdot \dfrac{5 \cdot 10^{6}}{mm^{3}} \cdot 2 \cdot (3.5 \mu m)^{2} \cdot \pi = \textcolor{red}{2309 m^{2}} $ 
        \item[f)] $ 5.67 \cdot 10^{-8} \dfrac{W}{m^{2} \cdot K^{4}} \cdot (11cm)^{2} \pi \cdot (420 K)^{4} = \textcolor{red}{67 W} $
    \end{enumerate}
\end{enumerate}
\end{document}