% Dokumentenklasse: 
%   - {article} : fuer Kurzberichte.
% Dokumentenklasseoptionen:
%   - [11pt]    : Schriftgroeße
%   - [a4paper] : Seitenformat DIN A4
%   - [twoside] : zweiseitig bedruckt
%   - [fleqn]   : linksbuendige Ausrichtung der Gleichungen
\documentclass[11pt, a4paper, twoside, fleqn]{article}
% Das Paket inputenc erlaubt die direkte Eingabe von Sonderzeichen wie zum Beispiel deutschen Umlauten, um deren Trennung zu ermoeglichen wird zudem das Paket fontenc mit eingebunden.
\usepackage[utf8]{inputenc}
\usepackage[T1]{fontenc}
% Hierbei wird ngerman als Option des Paketes babel gesetzt.
\usepackage[ngerman]{babel}
% Mit fullpage kann die Texthoehe und -breite sowie die Raender festlegen, dass die Seite fast voll ist.
\usepackage{fullpage}
\usepackage{amsmath, amssymb}
\usepackage{multicol}
% With xcolor, pages, the font, frames and fields can be set in the available colors.
\usepackage{xcolor}
\usepackage{tikz}
% Customizable Enumerates/Itemizes
\usepackage{enumitem}
\setlength{\mathindent}{0cm}
\setlength\parindent{0pt}
\setlength{\parskip}{0em}
\renewcommand{\baselinestretch}{1.5}
\title{Potenzen und Wurzeln} 
\author{Ioannis Christodoulakis}
\date{\today}
\begin{document}
% The command \selectlanguage{ngerman} changes the default language of the document.
\selectlanguage{ngerman}
\maketitle
% The statement ends the current page.
\newpage
% This command causes LaTeX to create a table of contents.
%\tableofcontents
% The statement ends the current page.
\newpage
%%%%%%%%%%%%%%%%%%%%%%%%%%%%%%%%%%%%%%%%%%%%%%%%%%%%%%%%%%%%%%%%%%%%%%%%%%%%%%%%%%%%%%%%%%%%%%%%%%%%
%%%%%%%%%%%%%%%%%%%%%%%%%%%%%%%%%%%%%%%%%%%%%%%%%%%%%%%%%%%%%%%%%%%%%%%%%%%%%%%%%%%%%%%%%%%%%%%%%%%%
\section{Übung Potenzen}

\begin{enumerate}[itemsep=3ex, leftmargin=*]
\item $ (-1)^{2n -1} = \textcolor{red}{-1} $
\item $ -10^{2} = \textcolor{red}{-100} $
% Minus sign must be placed before the fraction.
\item $ (-10)^{-3} = \textcolor{red}{\dfrac{1}{(-10)^{3}} = -\dfrac{1}{1000}} $
\item $((-2)^{3})^{2} = \textcolor{red}{64} $
\item $ (-1)^{-2n} = \textcolor{red}{\dfrac{1}{(-1)^{2n}} = 1} $
% \left and \right are Delimiter sizes
\item $ -\left(\dfrac{3}{2}\right)^{-4} = \textcolor{red}{-\left(\dfrac{2}{3}\right)^{-4} = -\dfrac{16}{81}} $
\item $ x^7 \cdot x^{-3} \cdot x^{-2} = \textcolor{red}{x^{7-3-2} = x^{2}} $
\item $ \dfrac{y^{3} \cdot x^{-2}}{(x \cdot y)^2} =  \textcolor{red}{\dfrac{y^3}{x^{2}\cdot y^{2} \cdot x^{2}} = \dfrac{y}{x^4}} $
\item $ \left(\dfrac{z^{-5}}{z^{-2}}\right)^{-2} = \textcolor{red}{\left(\dfrac{z^{2}}{z^{5}}\right)^{-2} = \left(\dfrac{z^{5}}{z^{2}} \right)^{2} = (z^3)^2 = z^6} $
\item $ \left(\dfrac{3}{5}\right)^4 \div \left(\dfrac{6}{25}\right)^{4} = \textcolor{red}{\left(\dfrac{3}{5} \cdot \dfrac{25}{6} \right)^{4} = \left (\dfrac{5}{2}\right)^{4} = \dfrac{625}{16}} $ 
\item $ a^{n-2} \cdot a^{1-n} =\textcolor{red}{a^{n-2+1-n} = \dfrac{1}{a}} $
\item $ (a-b)^{3} \cdot (b-a)^{-3} = \textcolor{red}{\dfrac{(a-b)^3}{(-1 \cdot(a-b))^{3}}= -1} $
\item $ b^{2x-1} \cdot b^{2x+1} \div b^{3x-1} = \textcolor{red}{b^{2x-1+2x+1-3x+1} = b^{x+1} = b \cdot b^{x}} $
\item $ \left(\dfrac{a^{2}}{b^{3}} \right)^{-2} \cdot \dfrac{5a^{3}}{2b^{2}} \cdot 2ab^{-4} = \textcolor{red}{\left(\dfrac{b^{3}}{a^2}\right)^{2} \cdot \dfrac{5a^3}{2b^2} \cdot \dfrac{2a}{b^{4}} = 5} $
\item $ \dfrac{4n^{-2} m^{4}}{5c^{2} x^{-3}} \div \dfrac{8m^{3}c^{-1}x}{15n^{-2}c} =\textcolor{red}{\dfrac{4m^{4}x^{3}}{5c^{2} \cdot n^{2}} \cdot \dfrac{15c \cdot c }{n^{2} \cdot 8m^{3}x} = \dfrac{3mx^{2}}{2n^{4}}} $
\item $ \dfrac{u^{2} - t^{2}}{2u^{2} + 4ut + 2t^{2}} = \textcolor{red}{\dfrac{(u-t) \cdot (u+t)}{2 \cdot (u+t) \cdot (u+t)} = \dfrac{u-t}{2 \cdot (u+t)}} $
\item $ (r+r^{-1})^2 - (r-r^{-1})^2 = \textcolor{red}{r^{2} + 2 + \dfrac{1}{r^{2}} - (r^{2} -2 + \dfrac{1}{r^{2}}) = 4} $
\item $ (2-p)^{3} = \textcolor{red}{8-12p+6p^{2}-p^3} $
\item $ \dfrac{a^{15} - a^{10}} {a^{5}} = \textcolor{red}{\dfrac{a^{5} \cdot (a^{10} - a^{5})}{a^{5}} = a^{10} - a^{5}} $
\item $ \dfrac{\dfrac{a^{2} - b^{2}}{ab + b^{2}}}{\dfrac{(a-b)^{2}}{ab^{2}}} = \textcolor{red}{\dfrac{(a+b) \cdot (a-b)}{b \cdot (a+b)} \cdot \dfrac{ab \cdot b}{(a-b) \cdot (a-b)} = \dfrac{ab}{a-b}}$ 
\end{enumerate}

\newpage

\section{Übung Wurzeln}
\begin{enumerate} [itemsep=3ex]
\item $ \dfrac{\sqrt{27}}{\sqrt{3}} = \textcolor{red}{3} $
\item $ \sqrt{\dfrac{4}{49}} = \textcolor{red}{\dfrac{2}{7}} $
\item $ \sqrt{\dfrac{b^{8}}{25c^{2}}} = \textcolor{red}{\dfrac{b^{4}}{5c}} $
\item $ \dfrac{\sqrt{a^{3}b}}{\sqrt{ab^{5}}} = \textcolor{red}{\dfrac{a}{b^{2}}} $
\item $ \sqrt{2ac} \cdot \sqrt{\dfrac{8a}{c}} = \textcolor{red}{4a} $
\item $ \sqrt{\dfrac{9m^{3}}{5n}} \div \sqrt{\dfrac{81m}{20n^{5}}} = \textcolor{red}{\dfrac{2}{3} m \cdot n^{2}} $
\item $ \sqrt[3]{8r^{6}t^{4}} = \textcolor{red}{2r^{2} \cdot t^{3} \sqrt[3]{t}} $
\item $\sqrt{a} \cdot \sqrt[3]{a^{2}} = \textcolor{red}{ a \cdot \sqrt[6]{a}}$
\item $ \dfrac{\sqrt{a} \cdot \sqrt[3]{a}}{\sqrt[6]{a}} = \textcolor{red}{\sqrt[3]{a^{2}}} $
\item $ \dfrac{\sqrt{x}}{x^{\frac{1}{3}}} = \textcolor{red}{\sqrt[6]{x}} $
\item $ \dfrac{y^{\frac{3}{4}} \cdot \sqrt[6]{y}}{y^{\frac{7}{12}}} = \textcolor{red}{\sqrt[3]{y}} $
\item $ (75x)^{\frac{1}{2}} \div (3x)^{\frac{1}{2}} = \textcolor{red}{5} $
\end{enumerate}

Nenner ohne Wurzel:

\begin{enumerate} [itemsep=3ex]\addtocounter{enumi}{12}
\item $ \dfrac{5}{\sqrt{3} + 1} = \textcolor{red}{\dfrac{5 \cdot \sqrt{3} - 5}{2}} $
\item $ \dfrac{a - b}{\sqrt{a - b}} = \textcolor{red}{\sqrt{a - b}} $
\item $ \dfrac{\sqrt{5} + \sqrt{2}}{\sqrt{5} - \sqrt{2}} = \textcolor{red}{\dfrac{7 +2 \cdot \sqrt{10}}{3}} $
\item $ 1000^{- \frac{1}{3}} = \textcolor{red}{\dfrac{1}{10}} $
\item $ c^{- \frac{2}{3}} \cdot \sqrt[6]{c} \cdot c^{\frac{1}{2}} = \textcolor{red}{1} $
\item $ \sqrt[3]{z \cdot \sqrt[4]{\dfrac{1}{z}}} = \textcolor{red}{\sqrt[4]{z}} $
\item $ (8x^{-9})^{\frac{1}{3}} = \textcolor{red}{\dfrac{2}{x^{3}}} $
\item $ \sqrt[3]{x\sqrt{x}} \cdot \dfrac{x^{\frac{1}{6}}}{\sqrt[3]{x}} = \textcolor{red}{\sqrt[3]{x}} $
\item $ \sqrt[5]{\sqrt[4]{z^{10}}}= \textcolor{red}{\sqrt{z}} $
\end{enumerate}

\end{document}
