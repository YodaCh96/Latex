% Dokumentenklasse: 
%   - {article} : für Kurzberichte.
% Dokumentenklasseoptionen:
%   - [11pt]    : Schriftgröße
%   - [a4paper] : Seitenformat DIN A4
%   - [twoside] : zweiseitig bedruckt
%   - [fleqn]   : linksbündige Ausrichtung der Gleichungen
\documentclass[11pt, a4paper, twoside, fleqn]{article}
% Das Paket inputenc erlaubt die direkte Eingabe von Sonderzeichen wie zum Beispiel deutschen Umlauten, um deren Trennung zu ermöglichen wird zudem das Paket fontenc mit eingebunden.
\usepackage[utf8]{inputenc}
\usepackage[T1]{fontenc}
% Hierbei wird ngerman als Option des Paketes babel gesetzt.
\usepackage[ngerman]{babel}
% Mit fullpage kann die Texthöhe und -breite sowie die Ränder festlegen, dass die Seite fast voll ist.
\usepackage{fullpage}
\usepackage{amsmath, amssymb}
\usepackage{multicol}
% With xcolor, pages, the font, frames and fields can be set in the available colors.
\usepackage{xcolor}
\usepackage{tikz}
% Customizable Enumerates/Itemizes
\usepackage{enumitem}
\setlength{\mathindent}{0cm}
\setlength\parindent{0pt}
\setlength{\parskip}{0em}
\renewcommand{\baselinestretch}{1.5}
\title{Repetitions Termumformungen} 
\author{Ioannis Christodoulakis}
\date{\today}
\begin{document}
% The command \selectlanguage{ngerman} changes the default language of the document.
\selectlanguage{ngerman}
\maketitle
% The statement ends the current page.
\newpage
% This command causes LaTeX to create a table of contents.
%\tableofcontents
% The statement ends the current page.
\newpage
%%%%%%%%%%%%%%%%%%%%%%%%%%%%%%%%%%%%%%%%%%%%%%%%%%%%%%%%%%%%%%%%%%%%%%%%%%%%%%%%%%%%%%%%%%%%%%%%%%%%
%%%%%%%%%%%%%%%%%%%%%%%%%%%%%%%%%%%%%%%%%%%%%%%%%%%%%%%%%%%%%%%%%%%%%%%%%%%%%%%%%%%%%%%%%%%%%%%%%%%%
\section{Repetitionsübung Termumformungen/ Potenzen/ Wurzeln}

Wenn nichts anderes vermerkt, sollen die Terme vereinfacht oder ausgerechnet werden.

\begin{enumerate}[itemsep=3ex , leftmargin=*]
\item $ -[2a - b \cdot (a - 2b) +ab] = \textcolor{red}{-[2a - ab + 2b^{2} +ab] = -2a - 2b^{2}} $ 
\item $ e \cdot (3e^{2} - 4) -2 \cdot (e^{3} - e^{2} - 2e) = \textcolor{red}{3e^{3} - 4e - 2e^{3} + 2e^{2} + 4e = e^{3} + 2e^{2}} $
\item $ \dfrac{-2x^{2} - 4xy - 2y^{2}}{x+y} = \textcolor{red}{\dfrac{-2 \cdot (x^{2} + 2xy + y^{2})}{x+y} = \dfrac{-2 \cdot(x + y) \cdot (x + y)}{(x + y)} = -2x - 2y} $
\item $ \dfrac{(a^{2} - 1) \cdot (2 - a) \cdot (-a^{2})}{a \cdot (a - 2) \cdot (a + 1)} = \textcolor{red}{\dfrac{(a + 1) \cdot (a - 1) \cdot (2 - a) \cdot (-1) \cdot a^{2}}{a \cdot (-1) \cdot (2 - a) \cdot (a + 1)} = a \cdot (a - 1)}  $
\item $ (s - t)^{3} = \textcolor{red}{1s^{3} - 3s^{2} \cdot t + 3st^{2} - t^{3}} $ \\
Nr. $ 6-12 $ : faktorisieren
\item $ ac - bd +bc - ad = \textcolor{red}{a \cdot (c - d) +b (c-d) = (c - d) \cdot (a + b)} $
\item $ x^{3} - x^{2} + y^{2} \cdot (x - 1) = \textcolor{red}{x^{2} \cdot (x - 1) + y^{2} \cdot (x - 1) = (x - 1) \cdot (x^{2} + y^{2}} $
\item $ 4x^{2} - 36y^{2} = \textcolor{red}{(2x + 6y) \cdot (2x - 6y)} $
\item $ -9u^{2} + 49z^{2} = \textcolor{red}{49z^{2} - 9u^{2} = (7z - 3u) \cdot (7z - 3u) } $
\item $ a^{2} - 9a + 20 = \textcolor{red}{(a- 5) \cdot (a - 4)} $
\item $ x^{2} + x - 42 = \textcolor{red}{(x + 7) \cdot (x - 6)} $
\item $ z^{4} - 1 = \textcolor{red}{(z^{2} + 1) \cdot (z^{2} = (z^{2} + 1) \cdot} (z + 1) \cdot (z - 1) $ \\ 
Nr. $ 13 -16:$ hier ist kürzem angesagt
\item $ \dfrac{ax + bx}{ay + by} = \textcolor{red}{\dfrac{x \cdot (a+b)}{y \cdot (a+b)} = \dfrac{x}{y}} $
\item $ \dfrac{a^{2} + 5a -24}{a^{2} - 5a + 6} = \textcolor{red}{\dfrac{(a + 8) \cdot (a - 3)}{(a - 2) \cdot (a-3)} = \dfrac{a + 8}{a - 2}} $
\item $ \dfrac{rs + r + s + 1}{rt + t +r + 1} = \textcolor{red}{\dfrac{r \cdot (s + 1) + 1 \cdot(s + 1)}{t \cdot(r + 1) +1 \cdot (r + 1)} = \dfrac{(s + 1) \cdot (r + 1)}{(r + 1) \cdot (t + 1) = \dfrac{s + 1}{t + 1}}} $
\item $ \dfrac{p^{3} - 2p^{2} + p}{pq - q} = \textcolor{red}{\dfrac{p \cdot (p^{2} - 2p + 1}{q \cdot (p - 1} = \dfrac{p \cdot (p - 1) \cdot (p - 1)}{q \cdot (p-1)} = \dfrac{p \cdot (p - 1)}{q}} $ 
\item $ \dfrac{6a}{5c} - \dfrac{11a}{15c} = \textcolor{red}{\dfrac{3 \cdot 6a}{3 \cdot 5c} - \dfrac{11a}{15c} = \dfrac{18a - 11a}{15}= \dfrac{7a}{15}} $
\item $ \dfrac{w^{2}}{w - 2} - w = \textcolor{red}{\dfrac{w^{2}}{w - 2} - w \cdot \dfrac{(w - 2)}{(w - 2)} = \dfrac{w^{2} - w^{2} + 2w}{w - 2} = \dfrac{2w}{w - 2}} $
\item $ \dfrac{a}{a + b} + \dfrac{b}{a - b} = \textcolor{red}{\dfrac{a}{(a + b)} \cdot \dfrac{(a - b)}{(a - b)} + \dfrac{b}{(a - b)} \cdot \dfrac{
(a + b)}{(a + b)} = \dfrac{ a^{2} - ab + ab + b^{2}}{(a + b) \cdot (a - b)} = \dfrac{a^{2} + b^{2}}{a^{2} - b^{2}}} $
\item $ \dfrac{3b^{2} - 3bc}{c} \div (6b - 6c) = \textcolor{red}{\dfrac{3b(b - c)}{c} \cdot \dfrac{1}{6 \cdot (b - c)} = \dfrac{3b}{6c} = \dfrac{b}{2c}} $

\end{enumerate}
























\end{document}